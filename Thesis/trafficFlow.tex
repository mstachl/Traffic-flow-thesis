\documentclass[11pt, oneside]{article}

%\usepackage{ngerman}
\usepackage{latexsym}
\usepackage{amssymb}
\usepackage{mathrsfs}
%\usepackage{psfig}
\usepackage{graphicx}
\usepackage[latin1]{inputenc}
\usepackage{abschlussarbeit}
\usepackage{hyperref}
\usepackage{verbatim}
\usepackage{mathtools}
\usepackage{dsfont}
\usepackage{ulem}
\usepackage{color}
\usepackage{natbib}

%\usepackage[T1]{fontenc}
%\usepackage[english]{babel}
%\usepackage[ngerman]{babel}
%\usepackage{lmodern}
%\usepackage{pict2e}
%\usepackage{amsmath, amssymb, amstext, amsfonts, mathrsfs}
%\usepackage[squaren]{SIunits}
%\usepackage[latin1]{inputenc}
%\usepackage[ngerman]{babel}
%\usepackage{pict2e}
%\usepackage{graphicx}
%\usepackage{xcolor}
%\usepackage{amssymb}
%\usepackage{amstext}
%\usepackage{amsmath}
%\usepackage{txfonts}
%\usepackage{amsfonts}
%\usepackage{mathrsfs}
%\usepackage{fancybox}
%\usepackage{framed}
%\usepackage{hyperref}
%\usepackage{dsfont}


\pagestyle{headings}
\bibliographystyle{plainnat}




%\newtheorem{Satz}{Satz}[chapter]
%% Ein sog. "Theorem" mit Abkuerzung "Satz" (das erste "Satz")
%% Das zweite "Satz" bezeichnet den Namen des Theorems. (z.B. "Satz x.y" erscheint im TeX-File).
%% [chapter] regelt die Numerierung der Saetze, in diesem Fall werden die Saetze pro Kapitel fortlaufend numeriert.
%
%\newtheorem{Korollar}[Satz]{Korollar}
%% Hier haben wir ein "Theorem" mit Abkuerzung "Korollar", welches im Tex-File als "Korollar" erscheint und in die "Theorem"-Nummerierung
%% fortlaufend eingebunden wird (das "[Theorem]" bewirkt dies).
%\newtheorem{Proposition}[Satz]{Proposition}
%% Abkuerzung ist "Proposition", Name ist "Proposition", wird in "Theorem"-Nummerierung eingebunden.
%%Lemma, ebenfalls in "Theorem"-Nummerierung eingebunden
%\newtheorem{Lemma}[Satz]{Lemma}
%%Definition, ebenfalls in "Theorem"-Nummerierung eingebunden
%\newtheorem{Definition}[Satz]{Definition}
%%Beispiel, ohne Nummerierung
%\newtheorem{Beispiel}{Beispiel}
%%Annahme, nach Kapiteln nummeriert
%\newtheorem{Annahme}{Annahme}[chapter]
%% Labelnummerierung in 'roemisch'.
%\renewcommand{\labelenumi}{(\roman{enumi})}

\begin{document}
\section{Introduction}
\section{Scalar conservation laws}
\subsection{Derivation of scalar conservation laws}
\subsection{Method of characteristics}
\subsection{Weak solutions}
\subsubsection{The Riemann problem}
\subsubsection{Shocks and rarefaction waves}
\subsubsection{Admissibility}
\section{Traffic flow modelling}
\subsection{The macroscopic Lighthill-Whitham-Richards model}
\subsection{Characteristic diagrams}
\subsection{The Riemann problem}
\section{Traffic flow on networks}
\subsection{Basic principles of networks}
\subsection{Traffic flow models on networks}
\section{Numerical approximation of conservation laws}
\subsection{The Godunov method}
\subsection{Examples}
\section{Experiments}
\subsection{Single buffer junction model}
\subsubsection{Experiment 1}
Let us consider a single junction of two incoming and two outgoing roads. Initially every road provides maximum flux density, namely $\rho_i=0.5$, smoothly on the entire space. The buffer zone offers maximum capacity $M=0.5$ and cars approaching the junction from both roads can cross it with equal priority value $c_i=0.5$. The values of M and c were chosen such that cars can enter the junction with maximum flux in case of an empty buffer and a smaller flux else (cf. \ref{eq:sbj}). The goal of the experiment is to see the evolution of the traffic densities over time in case of a non-equal distribution of the traffic at the junction. Therefore we assume a transition matrix $TM=\left(\begin{array}{cc}0.2 & 0.4 \\0.8 & 0.6\end{array}\right)$.
\newline
\newline
\underline{\textbf{Results:}} \newline
\begin{figure}[h]
\center
\includegraphics[scale=0.7]{2in2out}
\caption{Traffic distribution after $t=100s$.}
\end{figure}
\begin{figure}[h]
\center
\includegraphics[scale=0.5]{buffer05}
\caption{Buffer status over time up to $t=100s$.}
\end{figure}
More cars want to enter road 4 $\rightarrow$ road would be congested $\rightarrow$  buffer builds up over time. \newline Obviously the question arises if the total mass of cars on the network is preserved.
\begin{figure}[h]
\center
\includegraphics[scale=0.5]{totalmass}
\caption{Evolution of the total mass of traffic on the network up to time $t=100s$.}
\end{figure}
It is apparent that the total mass on the network is preserved, as long as the total influx equals the total outflux of the network. Considering incoming and outgoing roads as half-lines, the total mass is preserved. \newline
\newline
\underline{\textbf{Remark:}} \newline
As we can only consider a finite part of the network, we have to require the compatibility condition $\int_{\delta\Omega}f\cdot n \ ds = 0$ TRUE???? in order to preserve the amount of cars on the network. \newline \newline
A case where the compatibility condition is not satisfied can by easily constructed by taking the setting of Example 1 but leaving one outgoing road empty. In this case more cars are entering the network - in particular, the incoming flow equals $2f(0.5)$ - compared to the number of cars initially leaving the network - $f(0.5)$ from road 3, as road 4 is initially empty. This implicates an increase of the total traffic mass on the network (see figure \ref{fig:notpreserved}).
\begin{figure}[h]
\center
\includegraphics[scale=0.5]{totalmass_notpreserved}
\caption{\label{fig:notpreserved} Evolution of the total mass of traffic on the network for non-fulfilled compatibility condition.}
\end{figure}
\newpage
\section{Setting}
\subsection{Scalar conservation law}
Consider a single junction consisting of $m$ incoming and n outgoing roads. The corresponding index sets are $\mathcal{I}=\lbrace 1,...,m\rbrace$ and $\mathcal{O}=\lbrace m+1,...,m+n\rbrace$, respectively. On road $k$ the scalar conservation law is thus defined as:
\begin{equation*}
	\rho_t+f_k(\rho)_x = 0
\end{equation*}
In the infinite setting we have $x\in\left[-\infty,0\right]$ for incoming and $x\in\left[0;\infty\right]$ for outgoing roads, $t\geq 0$. \newline
The flux function on road $k$ is defined via $f_k(\rho )=\rho *v(\rho )$, where $v(\rho)$ denotes the average speed of cars on the road and $\rho\in\left[0,\rho^{jam}\right]$ the density of cars. \newline \newline
In order to solve the corresponding Riemann problem on the network, boundary conditions have to imposed. Thus we define \[\]
\section{Networks}
\subsection{Traffic flow on networks}
\begin{itemize}
	\item complex networks: \begin{itemize}
		\item  nodes $\mathcal{N}$ represent junctions
		\item roads $i\in\mathcal{I}, j\in\mathcal{O}$ represent roads \textbf{TODO: differenciation between in-and outgoing maybe later}		
		\item $A\in\mathbb{R^{(m+n)\times(m+n)}}$ adjacency matrix representing the network, e.g. $A=\left(\begin{array}{ccc}
		0 & 0 & 0  \\
		0 & 0 & 0 \\
		1 & 1 & 0  
			
		\end{array}\right) $ would represent the network displayed in \ref{fig:merge}. 
		\item In contrast to the adjacency matrix, transition matrices are defined only for single junctions. Let $J$ be a junction with $m$ incoming and $n$ outgoing roads. Then the transition matrix $TM_J\in\mathbb{R}^{n\times m}$ is defined by \[TM_J=\left(\begin{array}{ccc} P_{m+1,1} & ... & P_{m+1,m} \\
		\vdots & \ddots & \vdots \\
		P_{m+n,1} & \dots & P_{m+n,m} 
		\end{array}\right)\]
		From this adjacency matrix one can easily derive the transition matrices for every single junction. In the 			stated example one would conclude \[TM_J=\left(\begin{array}{cc} 1 & 1
		\end{array}\right)\] as the transition matrix representing the single merge.
		\begin{figure}[h]
	\includegraphics[scale=0.4]{merge}
	\caption{Simple merge of two roads \label{fig:merge}}
\end{figure}
	\item \textbf{Driver's preferences $\Theta_{ij}$} define the fraction of cars on road $i$ that want to exit the junction in direction of road $j$. \\
	\[\sum_j\Theta_{ij}=1\]
	\item \textbf{Relative priorities given to incoming roads} e.g. external effects like traffic lights \end{itemize}
\end{itemize}
\begin{itemize}
	
	\item Maximum fluxes on incoming roads: \[\omega_i(\rho):=\lbrace\begin{array}{ll}
		f(\rho) & \rho\leq\sigma \\
		f(\sigma) & \rho>\sigma
	\end{array}\]
	\item Maximum fluxes on outgoing roads: \[\omega_j(\rho):=\lbrace\begin{array}{ll}
		f(\sigma) & \rho\leq\sigma \\
		f(\rho) & \rho>\sigma
	\end{array}\]
	This property ensures that Riemann problems on incoming roads are solved by waves with negative speed, and Riemann problems on outgoing roads are solved by waves with positive speed.
\end{itemize}
\subsection{The Riemann problem on networks}
	
\section{Models}
\subsection{Single buffer junction model (SBJ)}
As the example before has shown there is no hope that the previous approach guarantees a unique solution for arbitrary initial conditions and assigned preference values $\Theta_{ij}$. We can circumvent this behavior by modelling a junction not as a fixed point, but instead as a buffer with fixed capacity that can "store" cars in a queue. \newline \newline
Consider a constant $M>0$, describing the maximum capacity of the junction at any given time, and constants $c_i>0, i\in\mathcal{I}$ accounting for priorities given to the differnt incoming roads. \newline
We then require that the incoming fluxes $\bar{f}_i$ satisfy
\[\label{eq:sbj}\bar{f}_i=\min\lbrace \omega_i,c_i(M-\sum_{j\in\mathcal{O}}q_j), \hspace{1cm} i\in\mathcal{I}\]
Furthermore, the outgoing fluxes $\bar{f}_j$ should satisfy
\[\bar{f}_j=\lbrace\begin{array}{ll}
	\omega_j & q_j>0 \\
	\min\lbrace\omega_j, \sum_{i\in\mathcal{I}}\bar{f}_i\bar{\Theta}_{ij}\rbrace & q_j=0

\end{array}\]
As seen, the outgoing fluxes are uniquely defined one the incoming fluxes are known in addition to the current status of the queue. \newline \newline
The queue thus changes its length by
	\begin{equation}
		\label{eq:q}		
		\dot{q}_j=\sum_{i\in\mathcal{I}}\bar{f}_i\bar{\Theta}_{ij}-\bar{f}_j
	\end{equation}
\textbf{Question:}
\begin{itemize}
	\item choice of M and c
	\item Discretization of \ref{eq:q} $\rightarrow$ difference quotient? 
\end{itemize}
\subsubsection{Simulation: preliminaries}
\begin{itemize}
	\item adjacency matrix
	\item inflow-outflow vectors
	\item capacity M for all junctions
	\item priorities $c_i$ for all outgoing roads for every junction
\end{itemize}
\subsubsection{Example: 2-2 junction}
Consider a junction with 2 incoming and 2 outgoing roads and the corresponding transition matrix $TM=\left(\begin{array}{cc} 0.4 & 0.2 \\ 0.6 & 0.8 
\end{array}\right)$ and inital densities $\rho_i(0,x)=0.5, i=1...4$. The queue has the following properties: $q(0)=0, M=1, c_3=c_4=0.5$. We also have incoming fluxes into road 1 and road 2 of $\rho_{in}=0.5$. 
\newline
\textbf{Numerical solution using Godunov scheme} \newline \newline
\begin{figure}[h]
\center
	\includegraphics[scale=0.6]{2in2out}
	\caption{Traffic distribution on the roads after t=100s. First two plots show the density on the incoming roads, the bottom two plots show the outgoing roads}
\end{figure}
\begin{figure}[h]
\center
	\includegraphics[scale=0.6]{Buffers}
	\caption{Evolution of the total queue length over time}
\end{figure}
\newline
Obervations:
\begin{itemize}
	\item jam emerges with idential speed for both incoming roads
	\item Buffer builds up in front of road 4, since the incoming flux is bigger than the maximum outgoing flux
\end{itemize}
\end{document}
